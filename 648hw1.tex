% Options for packages loaded elsewhere
\PassOptionsToPackage{unicode}{hyperref}
\PassOptionsToPackage{hyphens}{url}
%
\documentclass[
]{article}
\usepackage{amsmath,amssymb}
\usepackage{iftex}
\ifPDFTeX
  \usepackage[T1]{fontenc}
  \usepackage[utf8]{inputenc}
  \usepackage{textcomp} % provide euro and other symbols
\else % if luatex or xetex
  \usepackage{unicode-math} % this also loads fontspec
  \defaultfontfeatures{Scale=MatchLowercase}
  \defaultfontfeatures[\rmfamily]{Ligatures=TeX,Scale=1}
\fi
\usepackage{lmodern}
\ifPDFTeX\else
  % xetex/luatex font selection
\fi
% Use upquote if available, for straight quotes in verbatim environments
\IfFileExists{upquote.sty}{\usepackage{upquote}}{}
\IfFileExists{microtype.sty}{% use microtype if available
  \usepackage[]{microtype}
  \UseMicrotypeSet[protrusion]{basicmath} % disable protrusion for tt fonts
}{}
\makeatletter
\@ifundefined{KOMAClassName}{% if non-KOMA class
  \IfFileExists{parskip.sty}{%
    \usepackage{parskip}
  }{% else
    \setlength{\parindent}{0pt}
    \setlength{\parskip}{6pt plus 2pt minus 1pt}}
}{% if KOMA class
  \KOMAoptions{parskip=half}}
\makeatother
\usepackage{xcolor}
\usepackage[margin=1in]{geometry}
\usepackage{color}
\usepackage{fancyvrb}
\newcommand{\VerbBar}{|}
\newcommand{\VERB}{\Verb[commandchars=\\\{\}]}
\DefineVerbatimEnvironment{Highlighting}{Verbatim}{commandchars=\\\{\}}
% Add ',fontsize=\small' for more characters per line
\usepackage{framed}
\definecolor{shadecolor}{RGB}{248,248,248}
\newenvironment{Shaded}{\begin{snugshade}}{\end{snugshade}}
\newcommand{\AlertTok}[1]{\textcolor[rgb]{0.94,0.16,0.16}{#1}}
\newcommand{\AnnotationTok}[1]{\textcolor[rgb]{0.56,0.35,0.01}{\textbf{\textit{#1}}}}
\newcommand{\AttributeTok}[1]{\textcolor[rgb]{0.13,0.29,0.53}{#1}}
\newcommand{\BaseNTok}[1]{\textcolor[rgb]{0.00,0.00,0.81}{#1}}
\newcommand{\BuiltInTok}[1]{#1}
\newcommand{\CharTok}[1]{\textcolor[rgb]{0.31,0.60,0.02}{#1}}
\newcommand{\CommentTok}[1]{\textcolor[rgb]{0.56,0.35,0.01}{\textit{#1}}}
\newcommand{\CommentVarTok}[1]{\textcolor[rgb]{0.56,0.35,0.01}{\textbf{\textit{#1}}}}
\newcommand{\ConstantTok}[1]{\textcolor[rgb]{0.56,0.35,0.01}{#1}}
\newcommand{\ControlFlowTok}[1]{\textcolor[rgb]{0.13,0.29,0.53}{\textbf{#1}}}
\newcommand{\DataTypeTok}[1]{\textcolor[rgb]{0.13,0.29,0.53}{#1}}
\newcommand{\DecValTok}[1]{\textcolor[rgb]{0.00,0.00,0.81}{#1}}
\newcommand{\DocumentationTok}[1]{\textcolor[rgb]{0.56,0.35,0.01}{\textbf{\textit{#1}}}}
\newcommand{\ErrorTok}[1]{\textcolor[rgb]{0.64,0.00,0.00}{\textbf{#1}}}
\newcommand{\ExtensionTok}[1]{#1}
\newcommand{\FloatTok}[1]{\textcolor[rgb]{0.00,0.00,0.81}{#1}}
\newcommand{\FunctionTok}[1]{\textcolor[rgb]{0.13,0.29,0.53}{\textbf{#1}}}
\newcommand{\ImportTok}[1]{#1}
\newcommand{\InformationTok}[1]{\textcolor[rgb]{0.56,0.35,0.01}{\textbf{\textit{#1}}}}
\newcommand{\KeywordTok}[1]{\textcolor[rgb]{0.13,0.29,0.53}{\textbf{#1}}}
\newcommand{\NormalTok}[1]{#1}
\newcommand{\OperatorTok}[1]{\textcolor[rgb]{0.81,0.36,0.00}{\textbf{#1}}}
\newcommand{\OtherTok}[1]{\textcolor[rgb]{0.56,0.35,0.01}{#1}}
\newcommand{\PreprocessorTok}[1]{\textcolor[rgb]{0.56,0.35,0.01}{\textit{#1}}}
\newcommand{\RegionMarkerTok}[1]{#1}
\newcommand{\SpecialCharTok}[1]{\textcolor[rgb]{0.81,0.36,0.00}{\textbf{#1}}}
\newcommand{\SpecialStringTok}[1]{\textcolor[rgb]{0.31,0.60,0.02}{#1}}
\newcommand{\StringTok}[1]{\textcolor[rgb]{0.31,0.60,0.02}{#1}}
\newcommand{\VariableTok}[1]{\textcolor[rgb]{0.00,0.00,0.00}{#1}}
\newcommand{\VerbatimStringTok}[1]{\textcolor[rgb]{0.31,0.60,0.02}{#1}}
\newcommand{\WarningTok}[1]{\textcolor[rgb]{0.56,0.35,0.01}{\textbf{\textit{#1}}}}
\usepackage{graphicx}
\makeatletter
\newsavebox\pandoc@box
\newcommand*\pandocbounded[1]{% scales image to fit in text height/width
  \sbox\pandoc@box{#1}%
  \Gscale@div\@tempa{\textheight}{\dimexpr\ht\pandoc@box+\dp\pandoc@box\relax}%
  \Gscale@div\@tempb{\linewidth}{\wd\pandoc@box}%
  \ifdim\@tempb\p@<\@tempa\p@\let\@tempa\@tempb\fi% select the smaller of both
  \ifdim\@tempa\p@<\p@\scalebox{\@tempa}{\usebox\pandoc@box}%
  \else\usebox{\pandoc@box}%
  \fi%
}
% Set default figure placement to htbp
\def\fps@figure{htbp}
\makeatother
\setlength{\emergencystretch}{3em} % prevent overfull lines
\providecommand{\tightlist}{%
  \setlength{\itemsep}{0pt}\setlength{\parskip}{0pt}}
\setcounter{secnumdepth}{-\maxdimen} % remove section numbering
\usepackage{bookmark}
\IfFileExists{xurl.sty}{\usepackage{xurl}}{} % add URL line breaks if available
\urlstyle{same}
\hypersetup{
  pdftitle={648hw1},
  pdfauthor={Jinning Lin},
  hidelinks,
  pdfcreator={LaTeX via pandoc}}

\title{648hw1}
\author{Jinning Lin}
\date{2025-12-07}

\begin{document}
\maketitle

\section{Discuss the advantages and challenges associated with
Reading}\label{discuss-the-advantages-and-challenges-associated-with-reading}

One major advantage of an open data science approach is increased
transparency. When data and methods are shared openly, research becomes
easier to verify, question, and build upon. This can encourage
collaboration across disciplines and allow research findings to reach a
wider audience. However, open data science also faces real challenges.
Not all data can or should be made public, especially when it involves
sensitive personal or location-based information. Ethical concerns and
data misuse become serious risks.

The reading provides a useful example through the discussion of
large-scale voter data collected during U.S. election campaigns. If such
data were openly available, it could greatly benefit social research. At
the same time, it could raise serious privacy concerns and lead to
oversimplified interpretations if taken out of its political and social
context. This shows that open data science offers strong opportunities,
but only when combined with careful judgment and critical reflection.

\section{Import packages and read the
data}\label{import-packages-and-read-the-data}

\begin{Shaded}
\begin{Highlighting}[]
\FunctionTok{library}\NormalTok{(ggplot2)}
\FunctionTok{library}\NormalTok{(sf)}
\FunctionTok{library}\NormalTok{(tidyverse)}
\FunctionTok{library}\NormalTok{(ggspatial)}
\FunctionTok{library}\NormalTok{(viridis)}
\NormalTok{boulder }\OtherTok{\textless{}{-}} \FunctionTok{st\_read}\NormalTok{(}\StringTok{"E:/648{-}lab/BoulderSocialMedia/BoulderSocialMedia.shp"}\NormalTok{)}
\end{Highlighting}
\end{Shaded}

\begin{verbatim}
## Reading layer `BoulderSocialMedia' from data source 
##   `E:\648-lab\BoulderSocialMedia\BoulderSocialMedia.shp' using driver `ESRI Shapefile'
## Simple feature collection with 55519 features and 12 fields
## Geometry type: POINT
## Dimension:     XY
## Bounding box:  xmin: -788775 ymin: 1917813 xmax: -780555 ymax: 1930053
## Projected CRS: NAD_1983_Albers
\end{verbatim}

\begin{Shaded}
\begin{Highlighting}[]
\NormalTok{boulder}
\end{Highlighting}
\end{Shaded}

\begin{verbatim}
## Simple feature collection with 55519 features and 12 fields
## Geometry type: POINT
## Dimension:     XY
## Bounding box:  xmin: -788775 ymin: 1917813 xmax: -780555 ymax: 1930053
## Projected CRS: NAD_1983_Albers
## First 10 features:
##            id     DB   extent Climb_dist TrailH_Dis NatMrk_Dis Trails_dis
## 1  6517284333 Flickr 421678.2   1973.108   2368.567   2451.633   49.73422
## 2  6517281191 Flickr 421678.2   1973.108   2368.567   2451.633   49.73422
## 3  6517278961 Flickr 421678.2   1973.108   2368.567   2451.633   49.73422
## 4  6517276295 Flickr 421678.2   1973.108   2368.567   2451.633   49.73422
## 5  6517274727 Flickr 421678.2   1973.108   2368.567   2451.633   49.73422
## 6  6517272539 Flickr 421678.2   1973.108   2368.567   2451.633   49.73422
## 7  6517270109 Flickr 421678.2   1973.108   2368.567   2451.633   49.73422
## 8  6516904527 Flickr 421678.2   1973.108   2368.567   2451.633   49.73422
## 9  6516902971 Flickr 421678.2   1973.108   2368.567   2451.633   49.73422
## 10 6516900761 Flickr 421678.2   1973.108   2368.567   2451.633   49.73422
##    Bike_dis PrarDg_Dis PT_Elev Hydro_dis Street_dis                geometry
## 1  1437.134   1942.125    2064   1359.75   193.9165 POINT (-786099 1929916)
## 2  1437.134   1942.125    2064   1359.75   193.9165 POINT (-786099 1929916)
## 3  1437.134   1942.125    2064   1359.75   193.9165 POINT (-786099 1929916)
## 4  1437.134   1942.125    2064   1359.75   193.9165 POINT (-786099 1929916)
## 5  1437.134   1942.125    2064   1359.75   193.9165 POINT (-786099 1929916)
## 6  1437.134   1942.125    2064   1359.75   193.9165 POINT (-786099 1929916)
## 7  1437.134   1942.125    2064   1359.75   193.9165 POINT (-786099 1929916)
## 8  1437.134   1942.125    2064   1359.75   193.9165 POINT (-786099 1929916)
## 9  1437.134   1942.125    2064   1359.75   193.9165 POINT (-786099 1929916)
## 10 1437.134   1942.125    2064   1359.75   193.9165 POINT (-786099 1929916)
\end{verbatim}

\section{Add distance categories
column}\label{add-distance-categories-column}

\begin{Shaded}
\begin{Highlighting}[]
\NormalTok{boulder }\OtherTok{\textless{}{-}}\NormalTok{ boulder }\SpecialCharTok{\%\textgreater{}\%}
\FunctionTok{mutate}\NormalTok{(}\AttributeTok{Distance\_Category =} \FunctionTok{case\_when}\NormalTok{(}
\NormalTok{Street\_dis }\SpecialCharTok{\textless{}} \DecValTok{500} \SpecialCharTok{\textasciitilde{}} \StringTok{"near"}\NormalTok{,}
\NormalTok{Street\_dis }\SpecialCharTok{\textgreater{}=} \DecValTok{500} \SpecialCharTok{\&}\NormalTok{ Street\_dis }\SpecialCharTok{\textless{}=} \DecValTok{1000} \SpecialCharTok{\textasciitilde{}} \StringTok{"middle"}\NormalTok{,}
\NormalTok{Street\_dis }\SpecialCharTok{\textgreater{}} \DecValTok{1000} \SpecialCharTok{\textasciitilde{}} \StringTok{"far"}
\NormalTok{)) }
\end{Highlighting}
\end{Shaded}

\section{Display the distributions of points of different distance
categories on the
map}\label{display-the-distributions-of-points-of-different-distance-categories-on-the-map}

\begin{Shaded}
\begin{Highlighting}[]
\NormalTok{map\_plot }\OtherTok{\textless{}{-}} \FunctionTok{ggplot}\NormalTok{() }\SpecialCharTok{+}
  \FunctionTok{geom\_sf}\NormalTok{(}\AttributeTok{data =}\NormalTok{ boulder, }\FunctionTok{aes}\NormalTok{(}\AttributeTok{color =}\NormalTok{ Distance\_Category), }\AttributeTok{alpha =} \FloatTok{0.6}\NormalTok{) }\SpecialCharTok{+}
  \FunctionTok{theme\_minimal}\NormalTok{() }\SpecialCharTok{+}
  \FunctionTok{labs}\NormalTok{(}
    \AttributeTok{title =} \StringTok{"Shooting locations by road distance category in Boulder"}\NormalTok{,}
    \AttributeTok{color =} \StringTok{"Distance category"}
\NormalTok{  )}
\end{Highlighting}
\end{Shaded}

\section{Showing the spatial relationship between elevation and
distance}\label{showing-the-spatial-relationship-between-elevation-and-distance}

\begin{Shaded}
\begin{Highlighting}[]
\CommentTok{\#Boxplot of elevation vs. distance categories}
\NormalTok{box\_plot }\OtherTok{\textless{}{-}} \FunctionTok{ggplot}\NormalTok{(boulder, }\FunctionTok{aes}\NormalTok{(}\AttributeTok{x =}\NormalTok{ Distance\_Category, }\AttributeTok{y =}\NormalTok{ PT\_Elev)) }\SpecialCharTok{+}
  \FunctionTok{geom\_boxplot}\NormalTok{() }\SpecialCharTok{+}
  \FunctionTok{theme\_classic}\NormalTok{() }\SpecialCharTok{+}
  \FunctionTok{labs}\NormalTok{(}
    \AttributeTok{title =} \StringTok{"Elevation distribution of locations in different road distance categories"}\NormalTok{,}
    \AttributeTok{x =} \StringTok{"Distance category"}\NormalTok{,}
    \AttributeTok{y =} \StringTok{"Altitude (m)"}
\NormalTok{  )}
\end{Highlighting}
\end{Shaded}

\section{Map}\label{map}

\pandocbounded{\includegraphics[keepaspectratio]{648hw1_files/figure-latex/unnamed-chunk-4-1.pdf}}
\pandocbounded{\includegraphics[keepaspectratio]{648hw1_files/figure-latex/unnamed-chunk-4-2.pdf}}

\end{document}
